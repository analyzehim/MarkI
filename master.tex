\documentclass[18pt,a4paper]{article}
\usepackage[russian]{babel}
\usepackage{pgfplots}
\usepackage{amsmath}
\pgfplotsset{compat=1.9}
\usepackage[left=1cm,right=1cm,top=2cm,bottom=2cm]{geometry}
\usepackage{mathrsfs}
\usepackage{relsize}
\pgfplotsset{model/.style = {blue, samples = 100}}
\pgfplotsset{experiment/.style = {red}}
\usepackage{amsthm}
\usepackage{graphicx}
\usepackage{graphics}
\usepackage{amsfonts}
\usepackage[font=small,labelfont=bf]{caption}
\usetikzlibrary{calc,intersections} % Читайте мануал в Bonus/Books on TeX/Pictures for TeX/PGF, а также http://www.texample.net/tikz/examples/all/

\newtheorem{theorem}{Теорема}
\theoremstyle{plain}
\newtheorem{Th}{Теорема}[section]
\newtheorem{Lemma}{Лемма}[section]
\newtheorem{Cor}[Th]{Corollary}
\newtheorem{Prop}[Th]{Proposition}

 \theoremstyle{definition}
\newtheorem{Def}{Определение}[section]
\newtheorem{Conj}[Th]{Conjecture}
\newtheorem{Rem}[Th]{Remark}
\newtheorem{?}[Th]{Problem}
\newtheorem{Ex}[Th]{Example}
\begin{document}

  \begin{titlepage}
 \begin{center}
\includegraphics[width=0.5\textwidth]{msu.jpg}\\


    МОСКОВСКИЙ ГОСУДАРСТВЕННЫЙ УНИВЕРСИТЕТ\\*
    ИМЕНИ М.В.ЛОМОНОСОВА \\*



    \hrulefill
    \end{center}

    \begin{center}
   \Large Факультет вычислительной математики и кибернетики\\*
    \end{center}
\begin{center}
   \Large Кафедра исследования операций\\*
    \end{center}

    \vspace{10em}

    \begin{center}
    \Large Магистерская диссертация на тему:\\
    "Минимизация меры риска при формировании финансовых портфелей"
    \end{center}

    \vspace{10em}


\begin{flushright}
  \large
  \textbf{Выполнил:}\\
  Студент 611 группы\\
  Раев~Евгений~Олегович

  \vspace{5mm}

  \textbf{Научный руководитель:}\\
  к.ф.-м.н., доцент\\
  Морозов~Владимир~Викторович
\end{flushright}

    \vspace{\fill}

    \begin{center}
   Москва, 2017
    \end{center}

    \end{titlepage}
\fontsize{14pt}{20pt}\selectfont
\tableofcontents
\newpage

\centering\section{Введение}
\flushleft
В задачах из финансовой сферы, связанных с оптимизацией распределения ресурсов, как правило, присутствуют различные случайные факторы. Для оценки риска получить не тот доход, на который делался расчет, используются различные статистические характеристики.\\
Был введен критерий, который объединил в себе черты дохода и риска,и был назван VaR-критерием (Value-at-Risk). С математической точки зрения VaR-критерий представляет собой квантиль распределения случайных потерь.\\
Однако у этого критерия оказались свои недостатки, поскольку он не учитывает возможные критические ситуации, возникающие на "хвосте" распределения функции дохода.В качестве меры риска был предложен CVaR-критерий (Conditional Value-at-Risk), который как раз характеризует средние потери на "хвосте" распределения дохода. Но этот критерий никак не описывает возможный доход в благоприятных
случаях.\\
В начале статьи устанавливается связь критериев VaR и CVaR, записанных в терминах дохода и потерь, так как все результаты, связанные с этими критериями,
были получены ранее для целевой функции, имеющей смысл потерь.
Было проведено исследование когерентности данных критериев.\\
Далее, была предложена модель, описывающая процесс формирования портфеля, который далее меняет свою стоимость в течение нескольких периодов (при этом распределение ценных бумаг в портфеле неизменно).
На основе такой модели была предложена формула для минимизации VaR-критерия в терминах потерь.\\
Для двух- и трех- периодных моделей была осуществлена численная подстановка.

\centering\section{Основные определения}
\centering\subsection{Определения критериев VaR и CVaR в терминах функции доходности}
\flushleft\parindent=0.5cm

Для начала введем следующее определение:
\begin{Def} \label{main}
Функция доходности инвестора:
$$
X(x, R)
$$
\end{Def}

Для простоты обозначений будем предполагать, что в функцию $X(x, R)$ включен также начальный капитал $X_0$ инвестора.\\
Здесь вектор $x \in X \subset \mathbb{R}^m$ является оптимизируемой стратегией (управлением), а $R$ - вектор случайных параметров с реализациями $r \in \mathbb{R}^n$ и известной функцией распределения $G_R(r)$.\\
В силу случайности вектора $R$ оптимизация непосредственно целевой функции  $X(x,R)$ невозможна. Можно говорить лишь о максимизации некоторых статистических характеристик функции  $X(x,R)$. В дальнейшем будем предполагать. что функция дохода измерима по $r$ для каждой стратегии $x \in X$.\\
Назовем функцией вероятности при $x \in X$ вероятность такого события, что $X(x,R)$ будет не ниже некого наперед заданного уровня $x$:
\begin{Def} \label{main}
Функция вероятности для для функции доходности:

$$
P(u,x) =  \mathcal{P}\{R:X(x,R) \ge u\}.
$$
\end{Def}
Теперь, мы можем ввести следующие определения:\\

\begin{Def} \label{main}
VaR-критерий уровня $\alpha$ для функции доходности:
$$
u_\alpha(x)=max\{u:P(u, x)\ge \alpha\}.
$$
\end{Def}

VaR-критерий характеризует доход, который при выбранной стратегии $x$ с вероятностью, не меньшей $\alpha$, будет не меньше порога $u_\alpha(x)$.


\begin{Def} \label{main}

СVaR-критерий уровня $\alpha$ для функции доходности:
$$
CVaR: \psi_\alpha (x) = E[X | X \le u_\alpha (x)] = \frac{1}{1-\alpha} \int\limits_{0}^{\alpha} u dP(u,x)
$$
\end{Def}

СVaR-критерий характеризует средний доход в неблагоприятных случаях, возникающих с вероятностью $1-\alpha$, когда доход будет ниже значения VaR-критерия.\\

\begin{Th} \label{main} Можно выразить CVaR через VaR
\end{Th}
\begin{proof}
Действительно, если зафиксировать каждый $u$ как $u_\beta (x)$, то $P(u_\beta (x),x) = 1-\beta$, и тогда $d(P(u_\beta (x),x) = d(1-\beta) =  -d\beta$, следовательно \\
$$
\psi_\alpha (x) = \frac{1}{1-\alpha} \int\limits_{0}^{u_\alpha (x)} u dP(u,x) = \frac{1}{1-\alpha} \int\limits_{1}^{\alpha} u_\beta (x) (-d\beta) = \frac{1}{1-\alpha} \int \limits_{\alpha}^{1} u_\beta (x) d\beta
$$
\end{proof}




\centering\subsection{Определения критериев VaR и CVaR в терминах функции потерь}
\flushleft
Если $X_0$ - начальный капитал инвертора, то определим тогда функцию потерь:

\begin{Def} \label{main}
Функция потерь инвестора:
$$
L(x, R) = X_0 - X(x, R).
$$
\end{Def}

Тогда, соответственно:

\begin{Def} \label{main}
Функция вероятности для функции потерь:

$$
\tilde{P}(u, x) =  \mathcal{P}\{R:L(x,R) \le u\}.
$$
\end{Def}

\begin{Def} \label{main}
VaR-критерий уровня $\alpha$ для функции потерь:
$$
\tilde{u_\alpha}(x)=min\{u:\tilde{P}(u, x) \ge \alpha\}.
$$
\end{Def}
\begin{Def} \label{main}
СVaR-критерий уровня $\alpha$ для функции потерь:

$$
CVaR: \tilde{\psi_\alpha}(x) = E[L | L \ge \tilde{u}_\alpha (x)] = \frac{1}{1-\alpha} \int\limits_{0}^{\alpha} u d\tilde{P}(u, x)
$$
\end{Def}

СVaR-критерий характеризует средние потери в неблагоприятных случаях, возникающих с вероятностью $1-\alpha$, когда потери будут выше значения VaR-критерия.\\

\begin{Th} \label{main} Можно выразить CVaR через VaR
\end{Th}
\begin{proof}
Доказательство аналогично доказательству \textbf{Теоремы 2.1}:
$$
\tilde{\psi_\alpha}(x)=\frac{1}{1-\alpha}\int_\alpha^1 \tilde{u}_\beta(x)d\beta.
$$
\end{proof}

\centering\subsection{Связь критериев VaR и CVaR в терминах дохода и потерь}
\flushleft
Будем считать, что связь между функцией потерь и функцией дохода инвестора показана в Определении 2.5. Тогда, выполняются следующие зависимости:\\
\vspace{2pc}

\begin{Th} \label{main} Для любого $\alpha \in (0,1)$ верно:
$$
u_\alpha(u) = X_0 - \tilde{u}_\alpha(u)
$$
\end{Th}

\begin{proof}
Преобразуем функцию вероятности для дохода в функцию вероятность для потерь:
$$
P_x(u) =  \mathcal{P}\{R:X(x,R) \ge u\} =\mathcal{P}\{R:X_0 - L(x,R) \ge u \} =
$$
$$
= \mathcal{P}\{R:L(x,R) \le X_0 - u\} = \tilde{P}(u,x) .
$$
Тогда VaR-критерий для функции доходности примет вид:
$$
u_\alpha(u) = \max\{u:P_u(u)\ge \alpha  \} = \max\{u:\tilde{P}_{X_0-u}(u) \ge \alpha \}
$$
Сделаем замену переменных $\omega = - u$.\\
Получим, что
$$
u_\alpha(u) = \max\{u:\tilde{P}_{X_0-u}(u) \ge \alpha \} = - \min\{\omega:\tilde{P}_{X_0-\omega}(u) \ge \alpha  \}
$$
Сделаем еще одну замену переменных $\omega = X_0 - \tilde{u}$.\\
Тогда:
$$
u_\alpha(u) = - \min\{\omega:\tilde{P}_{X_0-\omega}(u) \ge \alpha  \} = X_0 - \min\{\tilde{u}:\tilde{P}_{\tilde{u}}\ge \alpha\} = X_0 - \tilde{u}_\alpha(u)
$$
Что и требовалось доказать.
\end{proof}
\vspace{2pc}
\begin{Th} \label{main} Для любого $\alpha \in (0,1)$ верно:
$$
\psi_\alpha(u) = X_0 - \tilde{\psi_\alpha}(u)
$$
\end{Th}

\begin{proof}
Пользуясь предыдущей теоремой, из Определения 2.4 получим:
$$
\psi_\alpha(u) =\frac{1}{1-\alpha}\int_\alpha^1 u_\beta(u)d\beta =\frac{1}{1-\alpha}\int_\alpha^1(X_0- \tilde{u_\beta}(u))d\beta=
$$
$$
=X_0 - \frac{1}{1-\alpha}\int_\alpha^1\tilde{u_\beta}(u)d\beta=X_0 - \tilde{\psi_\alpha}(u)
$$
Что и требовалось доказать.
\end{proof}


\centering\section{Альтернативные определения критериев VaR и CVaR в терминах функции потерь}

\centering\subsection{Определение квантильных функций и их свойств}
\flushleft


\begin{Def} \label{main}

$X$ - случайная величина дохода, $-X$ - случайная величина потерь.\\
\end{Def}


\begin{Def} \label{main}

$F(x) = P(X \le x) $ функция распределения\\
\end{Def}
Функция распределения непрерывна справа.\\
Зафиксируем некий $\alpha \in (0,1)$.
\begin{Def} \label{main}
Нижний $\alpha$-квантиль определен как:
$$
x_{\alpha} = q_\alpha(X) = \inf\{x\in \mathbb{R} :F(x) \ge \alpha\}
$$
\end{Def}
\begin{Def} \label{main}
Верхний $\alpha$-квантиль определен как:
$$
x_{(\alpha)} = q_\alpha(X) = \inf\{x\in \mathbb{R} :F(x) > \alpha\}
$$
\end{Def}

Эти квантили можно выразить через левую и правую обратную функцию на интервале (0,1)\\
\begin{Def} \label{main}
Если уравнение $F(x) = \alpha$ имеет корень, то левая обратная функция равна:\\
$$
F^{-1}_{L}(\alpha) = \min\{x|F(x)=\alpha  \}
$$
\end{Def}
Заметим, что минимум здесь действительно достигается, поскольку функция $F(x)$ непрерывна справа.\\
Если уравнение $F(x) = \alpha$ не имеет корня, то существует такое $x_0$, что
$$
F(x_0) > \alpha > F(x_0-0) = \lim\limits_{x \rightarrow x_0-0}F(x)
$$
Тогда,
$$
F^{-1}_{L}(\alpha) = x_0
$$
\center
\includegraphics[width=0.4\textwidth]{draw1.jpg}
\captionof{figure}{Поведение функции $F(x)$ }
\flushleft
При этому, будем говорить, что $\alpha$ соответствует точке разрыва (скачка) функции $F(x)$.\\

Аналогично определим $F^{-1}_{R}(\alpha)$ на интервале (0,1).\\
\begin{Def} \label{main}
Если уравнение $F(x) = \alpha$ имеет корень, то правая обратная функция равна:\\
$$
F^{-1}_{R}(\alpha) = \sup\{x|F(x)=\alpha  \}
$$
\end{Def}

Если уравнение $F(x) = \alpha$ не имеет корня, то существует такое $x_0$, что
$$
F(x_0) > \alpha > F(x_0-0) = \lim\limits_{x \rightarrow x_0-0}F(x)
$$

Тогда,
$$
F^{-1}_{R}(\alpha) = x_0
$$

Обратим внимание, что $sup$ здесь может не достигаться.\\
Например, в случае, изображенном на \bf Рис. 2,\rm\\
\center
\includegraphics[width=0.4\textwidth]{draw2.jpg}
\captionof{figure}{Поведение функции $F(x)$ }
\flushleft
множество $\{x|F(x) = \alpha_0 \} = [x_1, x_0]$ и $\sup\{x|F(x) = \alpha_0  \}$ не достигается.\\


\begin{Lemma} \label{main} Функция $F^{-1}_{L}(\alpha)$ не убывает и непрерывна слева на интервале (0,1). Кроме того, справедливы соотношения:
\begin{equation}
F^{-1}_{L}(F(x))\le x,  \forall x: F(x) \in (0,1)
\end{equation}
\begin{equation}
F(F^{-1}_{L}(\alpha)) \ge \alpha,  \forall \alpha \in (0,1)
\end{equation}
При этом, в (2) строгое неравенство выполнено в $\alpha$, соответствующим точкам скачка.
\end{Lemma}

\begin{proof}
\it Докажем монотонность функции $$F^{-1}_{L}(\alpha)$$\rm\\
Пусть $0 < \alpha < \beta < 1$. Если уравнение $F(x) = \alpha$ и $F(y) = \beta$ имеют решения $x$ и $y$, то необходимо, чтобы $x<y$, поскольку, в противном случае, когда $x\ge y$
и $F(x) = \alpha$ $\ge$ $F(y) = \beta$ (противоречие). Но из неравенства $x<y$ следует, что
$$
F^{-1}_{L}(\alpha)\le F^{-1}_{L}(\beta)
$$
Пусть уравнение $F(x) = \alpha$  не имеет корней. Тогда найдется такая точка $x_0$, для которой $F(x_0 -0) < \alpha < F(x_0)$. \\
Если при этом $F(x_0) \le \beta$ и $F(y) = \beta$ при некотором $y$, то $x_0 \le y$ и $x_0 = F^{-1}_{L}(\alpha) \le  F^{-1}_{L}(\beta) $.\\
Если $F(x_0) \le \beta$ и уравнение $F(y) = \beta$ не имеет корней, то найдется $y_0 > x_0$, для которой $F(y_0-0)<\beta<F(y_0)$ и $x_0 = F^{-1}_{L}(\alpha) < y_0 = F^{-1}_{L}(\beta).$\\
Монотонность доказана.\\

\hspace{20pt} \it Докажем неравенство (1).\rm\\
Пусть для точки $x$ выполнено $F(x) \in (0,1)$.\\
Положим $F(x) = \alpha$, и тогда, по определению, $F^{-1}_{L}(\alpha) \le x$\\
\hspace{20pt}\it Докажем неравенство (2).\rm\\
Если уравнение $F(x) = \alpha$ имеет корень, то, по определению, $F(F^{-1}_{L}(\alpha)) = \alpha$.\\
Если уравнение $F(x) = \alpha$ не имеет корня, то найдется точка $x_0$, такая, что $F(x_0-0) < \alpha < F(x_0)$ и $F^{-1}_{L}(\alpha)  = x_0$. Отсюда $F(x_0) = F(F^{-1}_{L}(\alpha)) > \alpha$\\
\hspace{20pt} \it Осталось проверить непрерывность слева для функции $F^{-1}_{L}(\alpha) $.\rm\\
Возьмем последовательность точек $\alpha_k < \alpha$, такую, что $\alpha_k \rightarrow \alpha$.\\
Тогда, в силу доказанной монотонности функции $F^{-1}_{L}(\alpha) $:\\
$$
F^{-1}_{L}(\alpha_k) < F^{-1}_{L}(\alpha), \forall k
$$
Докажем, что
$$
\lim\limits_{k \rightarrow \infty}F^{-1}_{L}(\alpha_k) = A = F^{-1}_{L}(\alpha)
$$
Предположим, что $A < F^{-1}_{L}(\alpha)$. Пусть уравнение $F(x)= \alpha$ имеет корень. Тогда, $F(F^{-1}_{L}(\alpha)) = \alpha$. Возьмем $x^{'} \in (A, F^{-1}_{L}(\alpha))$.\\
Имеем: $F^{-1}_{L}(\alpha_k) < x^{'} < F^{-1}_{L}(\alpha)$.\\
Тогда, $\alpha_k \le F(F^{-1}_{L}(\alpha_k)) \le F(x^{'}) \le F(F^{-1}_{L}(\alpha)) = \alpha$, $\forall k$.\\
Отсюда следует, что $F(x^{'}) = \alpha$, что противоречит неравенству $x^{'} < F^{-1}_{L}(\alpha)$.\\
Осталось рассмотреть случай, когда уравнение $F(x)= \alpha$ не имеет корня.\\
Тогда найдется точка $x_0$, такая, что:\\
$$
F(x_0 -0) < \alpha < F(x_0), F^{-1}_{L}(\alpha)=x_0
$$
Но при $\alpha_k$ близких к $\alpha$, уравнение $F(x)= \alpha_k$ не имеет корней.\\
Поэтому, $F^{-1}_{L}(\alpha_k)=x_0$ и $A = x_0 = F^{-1}_{L}(\alpha)$.\\
Лемма 3.1 доказана.
\end{proof}


Такой же набор утверждений есть и для правой обратной функции $F^{-1}_{R}(\alpha)$:\\
\begin{Lemma} \label{main} Функция $F^{-1}_{R}(\alpha)$ не убывает и непрерывна справа на интервале (0,1). Кроме того, справедливы соотношения:
\begin{equation}
F^{-1}_{R}(F(x))\ge x,  \forall x: F(x) \in (0,1)
\end{equation}
\begin{equation}
F(F^{-1}_{R}(\alpha)) \ge \alpha,  \forall \alpha \in (0,1)
\end{equation}
\end{Lemma}

\begin{proof}
Доказательство аналогично Лемме 1.
\end{proof}

Из определения нижнего квантиля, следует, что:\\
$$
x_{\alpha} = F^{-1}_{L}(\alpha)
$$
Чтобы доказать аналогичную формулу для верхнего квантиля:\\
$$
x^{\alpha} = F^{-1}_{R}(\alpha), \alpha \in (0,1)
$$
нам потребуется некоторые вспомогательные утверждения:\\

\begin{Lemma} \label{main}
Для верхнего квантиля $x^{\alpha}$ справедливы следующие соотношения:
\begin{equation}
x^{\alpha} = \sup \{ x| F(x) \le \alpha \}
\end{equation}
\begin{equation}
x^{\alpha} = \sup \{ x| P(X<x) \le \alpha \}
\end{equation}
\end{Lemma}

\begin{proof}
\it Докажем неравенство (5).\rm\\
Пусть $a = \sup \{x|F(x) \le \alpha \}$.\\
Возьмем любое $x$, удовлетворяющее неравенству $F(x) > \alpha$.\\
Тогда $x \ge a$. Действительно, если $x<a$, то по определению $a$ найдется такая точка $x_1$, что $x < x_1 < a$ и $F(x_1) \le \alpha$. Но тогда $F(x) \le F(x_1) \le \alpha < F(x) $ - противоречие.\\
Итак, из неравенства $F(x)>\alpha$ следует, что $x \ge a$.
Следовательно,\\
$$
x^{\alpha} = \inf \{ x| F(x) > \alpha \} \ge a $$
Покажем, что строгого неравенства здесь не может быть.\\
Действительно, если $x^{\alpha} > a$, то возьмем $x_1 \in (a, x^{\alpha})$. Из неравенства $x^{\alpha} > x_1$ следует, что $F(x_1) \le \alpha$ по определению $x^{\alpha}$. Но, из неравенства $x_1 >a$ следует, что $F(x_1)>\alpha$ по определению величины $a$. Получили противоречие.\\
\hspace{20pt} \it Докажем неравенство (6).\rm\\
Если при некоторых $x$: $F(x) \le \alpha$, то $P(X<x) \le P(X \le x) = F(x) = \alpha$ и $P(X < x) \le \alpha$.
Отсюда следует, что
$$
a \le b = \sup \{x| P(X<x) \le \alpha  \}
$$
Докажем, что здесь $a=b$.\\
В самом деле, если $a<b$, то возьмем любое $x_1\in(a,b)$. Тогда, по определению величин $a$ и $b$, получим неравенство:
$$
P(X\le x_1) > \alpha \ge P(X<x_1)
$$
Из этого неравенства следует, что точка $x_1$ является точкой разрыва функции $F(x)$. Но функция $F(x)$ имеет не более чем счетное число точек разрыва. Получили противоречией, которое доказывает равенство (4).\\
Лемма 2.3 полностью доказана.
\end{proof}


\begin{Lemma}
$x^{\alpha} = F^{-1}_{R}(\alpha)$.\\
\end{Lemma}

\begin{proof}
В самом деле, если уравнение $F(x)=\alpha$ имеет корень, то
$$
x^{\alpha} = \sup \{x|F(x) \le \alpha \} = \sup \{x|F(x) = \alpha \} = F^{-1}_{R}(\alpha)
$$
Если уравнение $F(x)=\alpha$ не имеет корней, то найдется такое число $x_0$, что $F(x_0-0)<\alpha<F(x_0)$. Поэтому, $x^{\alpha} = x_0 = F^{-1}_{R}(\alpha)$
\end{proof}

Из полученных представлений для квантилей $x_{\alpha} = F^{-1}_{L}(\alpha)$ и $x^{\alpha} = F^{-1}_{R}(\alpha)$ следует, что функции $x_{\alpha}$ и $x^{\alpha}$ могут иметь точки разрыва.
\begin{Th} \label{main}
Для того, чтобы $x_{\alpha}=x^{\alpha}$ необходимо и достаточно, чтобы уравнение $F(x)=\alpha$ имело не более одного корня.
\end{Th}
\begin{proof}
Предварительно заметим, что из определений
$$
x_{\alpha} = \inf \{x|F(x) \ge \alpha  \}
$$
\center
и
\flushleft
$$
x^{\alpha} = \inf \{x|F(x) > \alpha \}
$$

следует, что  $x_{\alpha} \le x^{\alpha}$\\
\hspace{20pt} \it Докажем достаточность:\rm\\
Пусть уравнение $F(x)=\alpha$ имеет не более одного корня.  Докажем, что $x_{\alpha}=x^{\alpha}$\\
Предположим, что уравнение $F(x)=\alpha$ не имеет корней.\\
Тогда найдется такое число $x_0$, что $F(x_0-0) < \alpha < F(x_0)$. Отсюда $x_{\alpha}=x^{\alpha}=x_0$\\
Предположим, что уравнение $F(x)=\alpha$ имеет единственный корень $x_1$.\\
Тогда,
$$
x_{\alpha} = \inf \{x|F(x) \ge \alpha \} = \inf \{x|F(x)=\alpha  \} = x_1
$$
Но, с другой стороны,
$$
x^{\alpha} = \sup \{x|F(x) \le \alpha \} = \sup\{x|F(x)=\alpha \} =x_1
$$
Следовательно, $x_{\alpha}=x^{\alpha}$. Достаточность доказана.\\
 \hspace{20pt} \it Докажем необходимость:\rm\\
Пусть $x_{\alpha}=x^{\alpha}$. Если уравнение $F(x) =  \alpha$ имеет более одного корня, то множество его корней образует отрезок $(x_{\alpha}, x^{\alpha})$ или полуинтервал  $[x_{\alpha}, x^{\alpha})$, что противоречит равенству $x_{\alpha}=x^{\alpha}$. Необходимость доказана.\\
\end{proof}

\centering\subsection{Определение VaR и CVaR как квантильных функций}
\flushleft
\begin{Def} \label{main}
Верхний VaR (Value at Risk) определяется, как:
$$
VaR^{\alpha} = VaR^{\alpha} (X) = -x^{\alpha}
$$
Тем самым он выражен через потери.\\
\end{Def}

\begin{Th} \label{main}
Справедлива формула:
\begin{equation}
VaR^{\alpha} (X) = q_{1-\alpha}(-X)
\end{equation}
\end{Th}
\begin{proof}
$$
VaR^{\alpha} (X) = -x^{\alpha} = - \sup\{x |P(X<x) \le \alpha  \}=\inf\{-x |P(-X>-x) \le \alpha  \}=
$$
\center
[замена $-x = y$]
\flushleft
$$
=\inf\{y |P(-X>y) \le \alpha  \}=\inf\{y |1-\alpha \le P(-X \le y)  \} = q_{1-\alpha}(-X)
$$
Теорема доказана.
\end{proof}
Формула (7) позволяет интерпретировать верхний VaR как вероятность того, что вероятность низких потерь (не првышающих $VaR^{\alpha}(X)$) будет не меньше $1-\alpha$. То есть, при малых $\alpha$ вероятность низких потерь близка к 1.\\

\begin{Def} \label{main}
Нижний VaR (Value at Risk) определяется, как:
$$
VaR_{\alpha} = VaR_{\alpha} (X) =  -x_{\alpha}
$$
\end{Def}
Для нижнего VaR также справедлива формула:
\begin{equation}
VaR_{\alpha} (X) = q^{1-\alpha}(-X)
\end{equation}
\begin{proof}
$$
VaR_\alpha = -x_{\alpha} = -\inf \{x| P(X \le x) \ge \alpha \} = \sup \{-x| P(-X \ge -x) \ge \alpha \} =
$$
\center
[замена $-x = y$]
\flushleft
$$
=\sup\{y|P(-X \ge y) \ge \alpha\}=\sup\{y |1 - P(-X < y) \ge \alpha  \} =
$$
$$
=\sup\{y |P(-X < y) \le 1-\alpha  \} = q^{1-\alpha}(-X)
$$
Теорема доказана.
\end{proof}
Введем понятия хвостовых условных мер риска (Tail Conditional Expectation):\\
\begin{Def} \label{main}
Нижнее среднее хвостовое определяется как:
$$
TCE_{\alpha} = TCE_{\alpha}(X) = -E[X|X \le x_{\alpha}] =
$$
$$
=-E[X| -X \ge -x_{\alpha}=VaR_{\alpha}] = -E[X| -X \ge VaR_{\alpha}]
$$
\end{Def}
По смыслу, $TCE_{\alpha}$ задает условные средние потери, превышающие $VaR_{\alpha}$.\\
Заметим, что
$$
E[X|X \le x_{\alpha}] = \int\limits_{-\infty}^{x^{\alpha}} \frac{x}{F(x_\alpha)}dF(x),
$$
где $\frac{x}{F(x_\alpha)}$, $x \in (-\infty, x_\alpha] $- условная функция распределения на $(-\infty, x_\alpha]$.\\
Аналогично определим верхнее среднее:
\begin{Def} \label{main}
$$
TCE^{\alpha} = TCE^{\alpha}(X) = -E[X|X \le x^{\alpha}] =
$$
$$
-E[-X| -X \ge -x^{\alpha}=VaR^{\alpha}] = -E[X| -X \ge VaR^{\alpha}]
$$
\end{Def}

Введем понятие хвостового среднего уровня $\alpha$ (Tail Mean):

\begin{Def} \label{main}
$$
\bar{x_{\alpha}} = TM_{\alpha}(X) = \frac{1}{\alpha}E[X]
$$
$$
-E[-X| -X \ge -x^{\alpha}=VaR^{\alpha}] = -E[X| -X \ge VaR^{\alpha}]
$$
\end{Def}


Введем функцию индикатора:
$$
\mathbf{1}_A(x) =
\left\{\begin{matrix}
1, &x \in A, \\
0, &x \notin A,
\end{matrix}\right.
$$


Введем понятие хвостового среднего уровня $\alpha$ (Tail Mean):

\begin{Def} \label{main}
$$
\bar{x}_{\alpha} = TM_{\alpha}(X) = \frac{1}{\alpha}\Big(E[X\mathbf{1}_{\{X\le x_\alpha\}}] +x_\alpha(\alpha-P[X\le x_\alpha])\Big)
$$
\end{Def}
Чтобы выяснить смысл этого определения, перепишем формулу $\bar{x}_{\alpha}$ в эквивалентном виде:\\
$$
E[X\mathbf{1}_{\{X\le x_\alpha\}}] = E[X\mathbf{1}_{\{X < x_{(\alpha)}\}}] + x_\alpha[P(X \le x_\alpha) - P(x <x_\alpha)] \Leftrightarrow
$$
$$
\bar{x}_{\alpha}  = \frac{1}{2} \Bigg( E[X\mathbf{1}_{\{X < x_{(\alpha)}\} }] + x_\alpha (\alpha - P(X<x_\alpha)) \Bigg)
$$


В точке $x_{\alpha}$ скачок берется до уровня $\alpha$.\\
\center
\includegraphics[width=0.4\textwidth]{draw3.jpg}
\captionof{figure}{}
\flushleft

В данном случае, функция условного распределения равна:\\
$$
F_\alpha(x) = \frac{F(x)\mathbf{1}_{\{x<x_\alpha\}} + \alpha\mathbf{1}_{\{x\ge x_\alpha\}}   }{\alpha}
$$


Если в точке $x_{\alpha}$ нет скачка функции $F(x)$, то $\alpha = F(x_{\alpha})$ и $\bar{x}_{\alpha} = TCE_\alpha$.\\
Введем понятие средних высоких потерь на уровне $\alpha$ (Expected Shortfall):
\begin{Def} \label{main}
$$
ES_\alpha(x) = - \bar{x}_{\alpha}
$$
\end{Def}

Для дальнейшего нам понадобится случайная величина:\\
$$
\mathbf{1}^\alpha_{\{X \le x\}} =
\left\{\begin{matrix}
\mathbf{1}_{\{X \le x\}}, &P(X=x)=0, \\
\mathbf{1}_{\{X \le x\}} + \frac{\alpha-P(X \le x)}{P(X=x)}\mathbf{1}_{\{X = x\}}, &P(X=x)>0,
\end{matrix}\right.
$$

\begin{Lemma}
Случайная величина $\mathbf{1}^\alpha_{\{X \le x\}}$ имеет следующие свойства:
\begin{equation}
\mathbf{1}^{(\alpha)}_{X\le x_{(\alpha)}} \in [0,1]
\end{equation}
\begin{equation}
E[\mathbf{1}^{(\alpha)}_{X\le x_{(\alpha)}} ] = \alpha
\end{equation}
\begin{equation}
\frac{1}{\alpha}E[X\mathbf{1}^{(\alpha)}_{X\le x_{(\alpha)}} ] = \bar{x}_{(\alpha)}
\end{equation}
\end{Lemma}

\begin{proof}
\hspace{20pt} \it Докажем (9):\rm\\
Пусть $P(X=x_\alpha)>0$, то есть функция $F(x)$ имеет скачок в точке $x_\alpha$.\\
Если уравнение $F(x_\alpha)=\alpha$ имеет решения, то $F(x_\alpha)=\alpha$ и
$$
\mathbf{1}^\alpha_{\{X \le x\}} = \mathbf{1}_{\{X \le x_\alpha\}}+\frac{\alpha-P(X \le x_\alpha)}{P(X=x_\alpha)}=\mathbf{1}_{\{X \le x_\alpha\}} \in [0,1]
$$
Если уравнение $F(x_\alpha)=\alpha$ не имеет решения, то $F(x_\alpha)-\alpha < P(X=x_\alpha)$, а $P(X=x_\alpha)=F(x_\alpha)-F(x_\alpha-0)$ - величина скачка. Поэтому, \\
$$
\frac{\alpha-P(X \le x_\alpha)}{P(X=x_\alpha)} = \frac{\alpha-F(x_\alpha)}{F(X=x_\alpha)} > -1
$$
и
$$
\mathbf{1}^\alpha_{\{X \le x_\alpha\}}=\mathbf{1}_{\{X \le x_\alpha\}}+\frac{\alpha-P(X \le x_\alpha)}{P(X=x_\alpha)}\mathbf{1}_{\{X = x_\alpha\}}
$$
Если $X<x_\alpha$, то $\mathbf{1}^\alpha_{\{X \le x_\alpha\}}=\mathbf{1}_{\{X = x_\alpha\}} = 1$.\\
Если $X=x_\alpha$, то $\mathbf{1}^\alpha_{\{X \le x_\alpha\}} = 1 + \frac{\alpha-P(X\le x_\alpha)}{P(X=x_\alpha)} >0$, следовательно, $\mathbf{1}_{\{X = x_\alpha\}} \in [0,1]$.\\
\hspace{20pt} \it Докажем (10):\rm\\
Если $P(X=x_\alpha)=0$, то $E[\mathbf{1}_{\{X\le x_{\alpha}\}} ] =P(X \le x_\alpha) = F(x_\alpha)=\alpha$, поскольку скачок отсутствует.\\
Если $P(X=x_\alpha)>0$, то $E[\mathbf{1}^\alpha_{\{X\le x_{\alpha}\}} ] = E[\mathbf{1}_{\{X\le x_{\alpha}\}} ] + \frac{\alpha-P(X \le x_\alpha)}{P(X=x_\alpha)} \cdot P(X=x_\alpha) = \alpha$.\\
\hspace{20pt} \it Докажем (11):\rm\\
Если $P(X=x_\alpha)>0$, то:\\
$$
\frac{1}{\alpha}E[X\mathbf{1}^\alpha_{X\le x_{\alpha}} ] = \frac{1}{\alpha} \Big( E[X\mathbf{1}_{X\le x_{\alpha}} ] +
x_\alpha \frac{\alpha-P(X \le x_\alpha)}{P(X=x_\alpha)}\cdot P(X=x_\alpha)\Big) =\bar{x}_{\alpha}
$$

Если $P(X=x_\alpha)>0$, то:\\
$F(x_\alpha) = P(X \le x_\alpha) = \alpha$ и $\frac{1}{\alpha}E[X\mathbf{1}^\alpha_{X\le x_{\alpha}} ] = \frac{1}{\alpha}E[X\mathbf{1}_{X\le x_{\alpha}} ] = \bar{x}_{\alpha} $
Лемма 3.5 доказана.\\
\end{proof}

\begin{Lemma}
Пусть $v$ - случайная величина, равномерно распределенная на $[0,1]$. Тогда, случайные величины $X_{(v)}$ и $X$ имеют одинаковые распределения.
\end{Lemma}
\begin{proof}
Пусть $X_{(v)} \le X$. Тогда, по \bf{леммe 2.1.}\rm получается, что
$$
v \le F(F^{-1}_L (v)) = F(X_{(v)}) \le F(x)
$$
Поэтому, $$\{X_{(v)} \le X \} \subset \{v \le F(x) \}
$$
Пусть $v\le F(X)$. Тогда,
$$
x_{(v)}= F(F^{-1}_L (v)) \le  F(F^{-1}_L (F(X)))\le X
$$
Последнее неравенство следует из \bf{леммы 3.2.}\rm\\
Итак, доказали, что $\{X_{(v)} \le X \} $ и $ \{v \le F(x) \} $ совпадают. Поэтому,
$$P(X_{(v)} \le x) = P(v \le F(x)) = F(X) = P(X \le x), $$ следовательно, \bf{лемма 3.6.}\rm доказана.
\end{proof}

\begin{Lemma}
Для хвостового среднего $\bar{x}_{(\alpha)} $ верно следующее представление:
\begin{equation}
\bar{x}_{(\alpha)} = \frac{1}{\alpha} \int\limits^\alpha_0 x_{(u)}du, \alpha \in (0,1)
\end{equation}
\end{Lemma}
\begin{proof}
Сначала заметим некоторые соотношения:\\
Если $v \le \alpha$, то $$x_{(v)} =  F(F^{-1}_L (v) \le  F(F^{-1}_L (\alpha) = x_{(\alpha)}        $$\\
Если $v > \alpha$, то $x_{(v)} \ge x_{(\alpha)}$. Поэтому, из $v>\alpha$ и $x_{(v)} \le x_{(\alpha)}$ следует, что $x_{(v)} = x_{(\alpha)} $.\\
Получим, что:\\
$$
\{x_{(v)} \le x_{(\alpha)} \} = \Big( \{v \le \alpha\} \cup  \{ x_{(v)} \le x_{(\alpha)} \} \Big) \cap \Big( \{v>\alpha\} \cup  \{ x_{(v)} \le x_{(\alpha)}  \} \Big) =
$$
$$
= \{v \le \alpha\} \cup \big( \{v>\alpha\} \cap  \{ x_{(v)} = x_{(\alpha)}  \} \big)
$$
Отсюда следует, что
$$
\{v \le \alpha\} = \{ x_{(v)} \le x_{(\alpha)}  \} \setminus  \big( \{v>\alpha\} \cap  \{ x_{(v)} = x_{(\alpha)}  \} \big)
$$
Поэтому,
$$
\int\limits^\alpha_0 x_{(u)}du = E[x_{(v)}\mathbf{1}_{\{ v \le \alpha \}}] = E[x_{(v)}\mathbf{1}_{\{ x_{(v)}  \le x_{(\alpha)}  \}}] - E[x_{(v)}\mathbf{1}_{\{v>\alpha\} \cap  \{x_{(v)}=x_{(\alpha)}\}}]
$$
Случайные величины $x_{(v)}$ и $X$ одинаково распределены, поэтому, в первом слагаемом, величину $x_{(v)}$ можно заменить на $X$. Во втором слагаемом, событие $x_{(v)} = x_{(\alpha)}$ эквивалентно
событию $v \in \Big[F(x_{(\alpha)}-0),F(x_{(\alpha)})\Big]$.\\
Поэтому,\\
$$
\int\limits^\alpha_0 x_{(u)}du = E[X \mathbf{1}_{\{X \le x_{(\alpha)} \}}] - x_{(\alpha)}(P(X \le x_{(\alpha)}) - \alpha) \leftrightarrow
$$
$$
\frac{1}{\alpha} \int\limits^\alpha_0 x_{(u)}du = \frac{1}{\alpha} \Big( E[X \mathbf{1}_{\{X \le x_{(\alpha)} \}}] - x_{(\alpha)}(P(X \le x_{(\alpha)}) - \alpha) \Big)
$$
Лемма 3.7 доказана.
\end{proof}

\centering\section{Когерентность VaR и ES}

\centering\subsection{Общее определение когерентности}
\flushleft
В данном параграфе будет показано, обладают ли функции VaR и CVaR когерентностью, или нет.\\
Для начала, вспомним определение когерентности:\\
\vspace{1pc}
\begin{Def} \label{main}
$f:V \rightarrow \mathbb{R}$, является когерентной мерой, если она обладает следующими свойствами:

1. Отрицательной монотонности, то есть, если $x,y \in V$ и $x\ge y$, то $f(x) \ge f(y)$,\\

2. Полуаддитивности, то есть, если $x,y,x+y \in V$, то $f(x+y) \le f(x) + f(y)$\\

3. Положительной однородности, то есть, если $r \in \mathbb{R}$, r>0 и $x, rx \in V$, то $f(rx) = rf(x)$ \\

4. Инвариантностью относительно сдвига, то есть, если  $r \in \mathbb{R}$ и $x \in V$, то $f(r+x) = f(x) - r$\\

\end{Def}
\vspace{1pc}
\centering\subsection{Когерентность VaR}
\flushleft
Вспомним определение VaR:
$$
VaR^\alpha(X) = -\inf\{x\in \mathbb{R} : P[X \le x] \ge \alpha\}
$$

\begin{Th} \label{main} $VaR^\alpha(X)$ - монотонная функция, то есть если $ X_1 \le X_2$, то и $ VaR^\alpha(X_1) \le VaR^\alpha(X_2)$
\end{Th}

\begin{proof}
Если $X_1\le X_2$, то $P[X_1 \le x] \ge P[X_2 \le x]$ для любого $x \in \mathbb{R}$. Если число $x$ принадлежит множеству $\{x\in \mathbb{R} : P[X_1 \le x] \ge \alpha\}$, то оно также принадлежит и множеству $\{x\in \mathbb{R} : P[X_2 \le x] \ge \alpha\}$, а вот обратное неверно. Это означает, что
$$
\inf\{x\in \mathbb{R} : P[X_1 \le x] \ge \alpha\} \ge \inf\{x\in \mathbb{R} : P[X_2 \le x] \ge \alpha\}
$$
Из этого вытекает, что $VaR^\alpha(X_1) \le VaR^\alpha(X_2)$, что и требовалось доказать.
\end{proof}

\begin{Th} \label{main} $VaR^\alpha(X)$ обладает свойством положительной симметричности, то есть если $ r >0$, то и $ VaR^\alpha(rX) =r VaR^\alpha(X)$
\end{Th}

\begin{proof}
Это следует из самого определения VaR:
$$
VaR^\alpha(rX) = -\inf\{x\in \mathbb{R} : P[rX \le x] \ge \alpha\} =  -r\inf\{\frac{x}{r}\in \mathbb{R} : P[X \le \frac{x}{r}] \ge \alpha\}
$$
$$
 = r VaR^\alpha(X)
$$
\end{proof}

\begin{Th} \label{main} $VaR^\alpha(X)$ обладает свойством инвариантности относительно сдвига, то есть если $r \in \mathbb{R}$, то $VaR^\alpha(r+X) = VaR^\alpha(X) - r$
\end{Th}

\begin{proof}
Это следует из самого определения VaR:
$$
VaR^\alpha(r+X) = -\inf\{x\in \mathbb{R} : P[r+X \le x] \ge \alpha\} = -\inf\{x\in \mathbb{R} : P[X \le x-r] \ge \alpha\} =
$$
$$
 = -(\inf\{x-r\in \mathbb{R} : P[X \le x-r] \ge \alpha\} + r) = VaR^\alpha(X) - r
$$
\end{proof}

На основе доказанных выше теорем можно сделать вывод, что VaR не обладает когерентностью (для него не выполняется условие субаддитивности).

\centering\subsection{Когерентность ES}


\flushleft


Так как VaR не обладает свойством полуаддитивности, проверим, обладает ли им ES:\\

\begin{Th} \label{main} $ES_\alpha(X)$ - когерентная мера.
\end{Th}

\begin{proof}

\hspace{20pt} \it Докажем отрицательную монотонность:\rm\\
Если $X \ge 0$,\\
то $\bar{x}_{(\alpha)} \ge 0$ и $-\bar{x}_{(\alpha)} = ES_\alpha(X) \le 0$


\hspace{20pt} \it Докажем субаддитивность:\rm\\

Обозначим $Z = X+Y$.
По определению $ES_\alpha (X) = -\bar{x}_{(\alpha)}$, а по \bf{лемме 3.5: }\rm $$\alpha\bar{x}_{(\alpha)} = E[X \mathbf{1}^{(\alpha)}_{\{X \le x_{(\alpha)}  \}}]$$\\
Поэтому, достаточно доказать, что $\alpha(\bar{z}_{(\alpha)} - \bar{x}_{(\alpha)} - \bar{y}_{(\alpha)}) \ge 0$

Получается, что:
$$
\alpha(ES^\alpha(X)+ES^\alpha(Y)-ES^\alpha(Z)) = E[Z\mathbf{1}^{(\alpha)}_{Z\le z_{(\alpha)}}-\mathbf{1}^{(\alpha)}_{X\le x_{(\alpha)}}-\mathbf{1}^{(\alpha)}_{Y\le y_{(\alpha)}}]=
$$
$$
= E[X(\mathbf{1}^{(\alpha)}_{Z\le z_{(\alpha)}} - \mathbf{1}^{(\alpha)}_{X\le x_{(\alpha)}})+Y(\mathbf{1}^{(\alpha)}_{Z\le z_{(\alpha)}} - \mathbf{1}^{(\alpha)}_{Y\le y_{(\alpha)}})]
$$

Достаточно показать, что первое слагаемое неотрицательно.\\
Доказательство неотрицательности второго производится аналогично.\\
Заметим, что:\\
$$
\mathbf{1}^{(\alpha)}_{\{X \le x_{(\alpha)}  \}}=
\left\{\begin{matrix}
0, &X > x_{(\alpha)}, \\
1, &X < x_{(\alpha)}.
\end{matrix}\right.
$$
Отсюда следует, что
$$
\mathbf{1}^{(\alpha)}_{\{Z \le z_{(\alpha)}  \}} - \mathbf{1}^{(\alpha)}_{\{X \le x_{(\alpha)}  \}}=
\left\{\begin{matrix}
\ge 0, &X > x_{(\alpha)}, \\
\le 0, &X < x_{(\alpha)}.
\end{matrix}\right.
$$
Отсюда получаем оценку:\\
$$
E[X(\mathbf{1}^{(\alpha)}_{Z\le z_{(\alpha)}} - \mathbf{1}^{(\alpha)}_{X\le x_{(\alpha)}})] =
$$

$$
=E\Big[X\mathbf{1}_{\{X > x_{(\alpha)}\}}\big(\mathbf{1}^{(\alpha)}_{\{Z \le z_{(\alpha)}  \}} - \mathbf{1}^{(\alpha)}_{\{X \le x_{(\alpha)}  \}}  \big) \Big]+
$$

$$
+E\Big[X\mathbf{1}_{\{X < x_{(\alpha)}\}}\big(\mathbf{1}^{(\alpha)}_{\{Z \le z_{(\alpha)}  \}} - \mathbf{1}^{(\alpha)}_{\{X \le x_{(\alpha)}  \}}  \big) \Big]+
$$

$$
+E\Big[X\mathbf{1}_{\{X = x_{(\alpha)}\}}\big(\mathbf{1}^{(\alpha)}_{\{Z \le z_{(\alpha)}  \}} - \mathbf{1}^{(\alpha)}_{\{X \le x_{(\alpha)}  \}}  \big) \Big] \ge
$$

$$
\ge x_{(\alpha)}E[\mathbf{1}^{(\alpha)}_{\{Z \le z_{(\alpha)}  \}} - \mathbf{1}^{(\alpha)}_{\{X \le x_{(\alpha)}  \}}]
=x_{(\alpha)}\Big(E[\mathbf{1}^{(\alpha)}_{\{Z \le z_{(\alpha)}  \}}] - E[\mathbf{1}^{(\alpha)}_{\{X \le x_{(\alpha)}  \}}]\Big) =x_{(\alpha)}\Big(\alpha - \alpha\Big) = 0
$$
Субаддитивность доказана.
\hspace{20pt} \it Докажем положительную однородность:\rm\\
Пусть $h>0$. Тогда:\\
$$
(hx)_{(\alpha)} = \inf\Big\{y|P(hX \le y) \ge \alpha  \Big\} = h \cdot \inf\Big\{\frac{y}{h}|P(X \le \frac{y}{h}) \ge \alpha  \Big\}=
$$
\center
$\Big[$ Замена $x = \frac{y}{h}$ $\Big]$
\flushleft
$$
=h\cdot \inf\Big\{x|P(X \le x) \ge \alpha  \Big\} = h x_{(\alpha)}
$$
Отсюда, используя \bf{лемму 3.7}\rm, находим:
$$
\bar{(hx)}_{(\alpha)} = \frac{1}{\alpha}\int\limits^\alpha_0 (hx)_{(u)}du = \frac{h}{\alpha}\int\limits^\alpha_0 x_{(u)}du = h \cdot \bar{x}_{(\alpha)}
$$
Положительная однородность доказана.\\
\hspace{20pt} \it Докажем инвариантность относительно сдвига:\rm\\
Пусть $a$ - любое число. Тогда:\\
$$
(x+a)_{(\alpha)} = \inf \Big\{x|P(X+a \le x) \ge \alpha  \Big\} = \inf \Big\{-a|P(X \le x-a) \ge \alpha  \Big\} +a = \bar{x}_{(\alpha)} + a
$$
Отсюда:\\
$$
ES_\alpha(X+a) = -\overline{(x+a)}_{(\alpha)} = - \overline{x}_{(\alpha)} - a = ES_\alpha (X) - a
$$
Из доказанных выше утверждений следует когерентность $ES_\alpha (X).$\\
Теорема доказана.
\end{proof}

\centering\section{Дополнительные свойства меры риска $ES_\alpha(X)$}
\flushleft

\begin{Lemma}
Справедливо неравенство:
\begin{equation}
E\Big[X(\alpha\mathbf{1}^{(\alpha + \varepsilon)}_{\{X \le x_{(\alpha + \varepsilon)}  \} } - (\alpha + \varepsilon)\mathbf{1}^{(\alpha)}_{\{X \le x_{(\alpha)}  \} } )\Big] \ge
x_{(\alpha)}E\Big[\alpha\mathbf{1}^{(\alpha + \varepsilon)}_{\{X \le x_{(\alpha + \varepsilon)}  \} } - (\alpha + \varepsilon)\mathbf{1}^{(\alpha)}_{\{X \le x_{(\alpha)}  \} }   \Big]
\end{equation}
\end{Lemma}

\begin{proof}
Пусть $P(X=x_{(\alpha)})=0,P(X=x_{(\alpha + \varepsilon)})=0 $, тогда:\\
$$
E\Big[X(\alpha\mathbf{1}^{(\alpha + \varepsilon)}_{\{X \le x_{(\alpha + \varepsilon)}  \} } - (\alpha + \varepsilon)\mathbf{1}^{(\alpha)}_{\{X \le x_{(\alpha)}  \} } )\Big]
= E\Big[X(\alpha\mathbf{1}_{\{X \le x_{(\alpha + \varepsilon)}  \} } - (\alpha + \varepsilon)\mathbf{1}_{\{X \le x_{(\alpha)}  \} } )\Big] =
$$

$$
= E\Big[X(\alpha\mathbf{1}_{\{x_{(\alpha)} < X \le x_{(\alpha + \varepsilon)}  \} } - \varepsilon\mathbf{1}_{\{X \le x_{(\alpha)}  \} } )\Big] \ge
x_{(\alpha)}E\Big[X(\alpha\mathbf{1}_{\{x_{(\alpha)} < X \le x_{(\alpha + \varepsilon)}  \} } - \varepsilon\mathbf{1}_{\{X \le x_{(\alpha)}  \} } )\Big] =
$$
$$
= x_{(\alpha)}E\Big[X(\alpha\mathbf{1}^{(\alpha+\varepsilon)}_{\{ X \le x_{(\alpha + \varepsilon)}  \} } - (\alpha+\varepsilon)\mathbf{1}^{(\alpha)}_{\{X \le x_{(\alpha)}  \} } )\Big].
$$
Пусть $P(X=x_{(\alpha)})=0,P(X=x_{(\alpha + \varepsilon)})>0 $, тогда:\\
$$
E\Big[X(\alpha\mathbf{1}^{(\alpha + \varepsilon)}_{\{X \le x_{(\alpha + \varepsilon)}  \} } - (\alpha + \varepsilon)\mathbf{1}^{(\alpha)}_{\{X \le x_{(\alpha)}  \} } )\Big]
=
$$
$$
E\Big[X\alpha(\mathbf{1}_{\{X \le x_{(\alpha + \varepsilon)}  \} }+
\frac{\alpha+\varepsilon - P(X \le x_{(\alpha+\varepsilon)})}{P(X = x_{(\alpha+\varepsilon)})}\mathbf{1}_{\{X=x_{(\alpha+\varepsilon)}\}}) - (\alpha + \varepsilon)\mathbf{1}_{\{X \le x_{(\alpha)}  \} } \Big]=
$$
$$
= E\Big[X\Big\{\alpha\big(\mathbf{1}_{\{x_{(\alpha)} < X < x_{(\alpha + \varepsilon)}  \} }+ \mathbf{1}_{ \{ X = x_{(\alpha + \varepsilon)}  \} } +
\frac{\alpha+\varepsilon - P(X \le x_{(\alpha+\varepsilon)})}{P(X = x_{(\alpha+\varepsilon)})}\mathbf{1}_{\{X=x_{(\alpha+\varepsilon)}\}}\big) - \varepsilon\mathbf{1}_{\{X \le x_{(\alpha)}  \}}\Big\}\Big] \ge
$$
$$
\ge x_{(\alpha)} E\Big[\alpha\big(\mathbf{1}_{\{x_{(\alpha)} < X < x_{(\alpha + \varepsilon)}  \} }+ \mathbf{1}_{ \{ X = x_{(\alpha + \varepsilon)}  \} } +
\frac{\alpha+\varepsilon - P(X \le x_{(\alpha+\varepsilon)})}{P(X = x_{(\alpha+\varepsilon)})}\mathbf{1}_{\{X=x_{(\alpha+\varepsilon)}\}}\big) - \varepsilon\mathbf{1}_{\{X \le x_{(\alpha)}  \}}\Big] =
$$
$$
= x_{(\alpha)} E\Big[ \alpha\mathbf{1}^{\{\alpha+\varepsilon\}}_{X \le x_{(\alpha+\varepsilon)}} - (\alpha+\varepsilon)\mathbf{1}^{\alpha}_{\{X \le x_{(\alpha)}\}}    \Big]
$$
Аналогично разбираются и случаи:\\
$$
P(X=x_{(\alpha)})>0,P(X=x_{(\alpha + \varepsilon)})>0
$$
$$
P(X=x_{(\alpha)})>0,P(X=x_{(\alpha + \varepsilon)})=0
$$
Лемма 5.1. доказана.

\end{proof}

\centering\section{Постановка задачи}
\flushleft

\centering\subsection{Общие определения}
\flushleft
\noindent Будем считать, что на рынке есть два вида бумаг - безрисковые, и рисковые. Пусть $r$ - доход безрисковой бумаги, а $\xi \in [-1, \alpha]$ - максимальное возможное значение доходности рисковой бумаги. Также, $\alpha > r$\\

Тогда, если X(i) - стоимость портфеля в единицу времени, то доходность:
$$
\xi = \frac{X(1)-X(0)}{X(0) }\ge -1
$$

Будем считать, что если $X(1)=0$, то бумага обанкротилась.\\
\vspace{1em}

Рассматривается 2-периодная модель. Пусть $X_0$ - начальная стоимость портфеля, тогда $y_0 + y_1 = X_0$ - сумма стоимостей безрисковой ($y_0$) и рисковой ($y_1$) бумаг. Также считаем, что короткая позиция запрещена. Получаем $x_0 = \frac{y_0}{X_0}$ и $x_1 = \frac{y_1}{X_0}$ - относительная доля бумаг в деньгах.\\

\vspace{5mm}
Если 2 периода, то:\\
1) $-1\le \xi_1 \le \alpha_1$ - доходность рисковой бумаги за первый период\\
2) $-1\le \xi_2 \le \alpha_2$ - доходность рисковой бумаги за второй период\\
\vspace{3mm}
Средняя доходность: $M_i>r>0, i =1,2$\\
Математическое ожидание доходности: $m_i = E\xi_i, i =1,2$\\
\vspace{3mm}
Получаем, что $X_1 = X_0(1+rx_0+\xi_1 x_1) $ - стоимость портфеля после первого периода, а $X_2 = X_0(1+rx_0+\xi_1 x_1)(1+rx_0+\xi_2 x_1) $ - стоимость портфеля после первого периода, с учетом того, что доли бумаг не меняются\\
\vspace{3mm}
Если доля бумаг будет меняться, то каждый период независим от предыдущего, следовательно можно рассматривать как последовательность 1-периодных моделей. В рамках данной работы доля бумаг в портфеле не будет меняться.\\
\vspace{3mm}
$$
X_n = X_0 \prod\limits_{i=1}^n (1+rx_0+m_i x_1)
$$
\vspace{3mm}
Если $m_i > r$, то, надо вкладывать в рисковые бумаги.\\

Определим теперь доходность в терминах потерь:
$$
L(x,R) = X_0 - X_n
$$

\centering\subsection{Парадокс разорения}
\flushleft

\begin{Th} \label{main}
Если вложить весь капитал в рисковые бумаги, то на большом промежутке времени, вероятность разорения становится близкой к 1.
\end{Th}
\begin{proof}
Рассмотрим равномерное распределение величины $\xi$.\\
$P(\xi_i = -1) = \varepsilon >0$ - вероятность разорения.\\
Пусть $A_i = \xi_i = -1$ - это событие разорения.\\
Рассмотрим, какая будет вероятность разорения через n шагов \\
Пусть $B_i=A_i\cap \prod\limits_{j=1}^{i-1}A_j$ - вероятность разорения за n шагов.\\
Тогда,
$$
P(\cup_{i=1}^n B_i) = \sum\limits_{i=1}^n P(B_i) = \sum P(A_i) \prod\limits_{j=1}^{i-1}A_j = \sum\limits_{i=1}^n \varepsilon (1-\varepsilon)^{i-1}
$$
И, в свою очередь, $\sum\limits_{i=1}^n (1-\varepsilon)^{i-1} \longrightarrow \frac{1}{1-(1-\varepsilon)} = \frac{1}{\varepsilon}$, как убывающая геометрическая последовательность, следовательно:\\
$$
\sum\limits_{i=1}^n \varepsilon (1-\varepsilon)^{i-1}  \longrightarrow \varepsilon \cdot \frac{1}{\varepsilon} = 1
$$
Вероятность разорения равна 1, при большом n.
\end{proof}

\centering\subsection{VaR и CVaR}
\flushleft
Введем новые определения.\\
$x = x(x_0,x_1)$.\\
Пусть $\tilde{P}(\tilde{u},x) = P(L(x,R) \le \tilde{u})$ - функция распределения стоимости портфеля\\
Будем считать, что некая вероятность $\alpha$ задана.\\
Тогда,\\
\vspace{2mm}
$$
VaR: \tilde{u}_\alpha (x) = \min\limits_{x} \{\tilde{P}(\tilde{u},x)\ge \alpha\}
$$
Функция $\tilde{P}$ возрастающая и непрерывная, то есть можно считать,что
$$
\tilde{u}_\alpha(x) = \min\limits_{x} \{\tilde{P}(\tilde{u},x) = \alpha\}
$$

Таким образом, необходимо минимизировать $\tilde{u}_\alpha (x)$.\\
Теперь определим CVaR:\\
$$
CVaR: \psi_\alpha (x) = E[L_2 | L_2 \ge \tilde{u}_\alpha (x)] = \frac{1}{1-\alpha} \int\limits_{0}^{\alpha} \tilde{u} dF(x,\tilde{u})
$$

Можно выразить CVaR через VaR. Если сопоставить каждому $\tilde{u}$ в соответствие $\tilde{u}_\gamma (x)$, то $F(x,\tilde{u}_\alpha^c (x)) = 1-\gamma$, и тогда $d(F(x,\tilde{u}_\alpha^c (x)) = d(1-\gamma) =  -d\gamma$, следовательно \\
$$
\tilde{u}_\alpha (x) = \frac{1}{1-\alpha} \int\limits_{0}^{\tilde{u}_\alpha (x)} \tilde{u} dF(x,\tilde{u}) = \frac{1}{1-\alpha} \int\limits_{1}^{\alpha} \tilde{u}_\gamma (x) (-d\gamma) = \frac{1}{1-\alpha} \int \limits_{\alpha}^{1} \tilde{u}_\gamma (x) d\gamma
$$

\centering\section{Минимизация VaR в двухпериодной модели}
\flushleft
Рассмотрим задачу минимизации VaR:
$$
\tilde{u}_\alpha (x) -> \min\limits_x
$$
Предположим, что $\xi_1 \in [-1, a_1]$ и $\xi_2 \in [-1, a_2]$ - равномерно распределенные случайные величины.\\
Также, $X_0 = 1$
Пусть у нас двухпериодная модель, то есть:
$$
X_2 = (1+rx_0+\varepsilon_1 x_1)(1+rx_0+\xi_2 x_1)
$$
Тогда,
$$
L_2 = 1-(1+rx_0+\varepsilon_1 x_1)(1+rx_0+\xi_2 x_1)
$$
Рассмотрим функцию
$$
\tilde{F}(x,\tilde{u}_\alpha (x) ) = P(L_2 \le \tilde{u}) = P\big(1-(1+rx_0+\xi_1 x_1)(1+rx_0+\xi_2 x_1\big) \le \tilde{u})
$$

1. Если $x_1 = 0$, то:\\
$$
\tilde{F}(x,\tilde{u}) =
\left\{\begin{matrix}
1, &(1+r)^2 \ge \tilde{u}-1 \\
0, &(1+r)^2 < \tilde{u}-1,
\end{matrix}\right.
$$
2. Если $x_1 \neq0$, то\\

\includegraphics[width=1\textwidth]{graph.jpg}
Рассчитаем $\xi_1^*$:
$$
1- (1+rx_0+\xi_1^* x_1)(1+rx_0- x_1) = \tilde{u}
$$
То есть,
$$
\xi_1^* = \frac{\frac{1-\tilde{u}}{1+rx_0-x_1}-1-rx_0}{x_1}
$$
$$
\tilde{F}(x,\tilde{u}) = P(L_2\le \tilde{u})=\frac{mes(S^+)}{mes} = 1-\frac{mes(S^-)}{mes}
$$
Учтем также, что:\\
$$
\xi_1^* \le a_1 \Leftrightarrow \frac{1}{x_1}[\frac{1-\tilde{u}}{1+rx_0-x_1} -1-rx_0]\le a_1\Leftrightarrow
$$
$$
\Leftrightarrow 1-\tilde{u} \le(1+rx_0-x_1)(1+rx_0+a_1 x_1)
$$
Аналогично и для $\xi_2^*$, тогда, в итоге, получается следующее ограничение:\\
$$
1-\tilde{u} \le(1+rx_0-x_1)(1+rx_0+\min(a_1,a_2) x_1)
$$
\vspace{15pt}
Нам необходимо $\tilde{u}_\alpha (x)$, как корень уравнения:\\
$$
F(x,\tilde{u}_\alpha (x)) = \alpha
$$


Найдем $mes(S^{-1})$:
$$
mes(S^-)=\int\limits_{-1}^{\xi^*} (\frac{1}{x_1}(\frac{1 - \tilde{u}}{1+rx_0+\xi_1 x_1}-1-rx_0)+1)) d\xi_1 =
$$
$$
=\frac{1 - \tilde{u}}{x_1^2}ln(\frac{1 - \tilde{u}}{(1+rx_0-x1)^2}-\frac{1 - \tilde{u}}{x_1^2}+\frac{(1+rx_0-x_1)^2}{x_1^2}
$$
Таким образом,
$$
F(x,\tilde{u}_\alpha (x)) = 1 -\frac{1 - \tilde{u}}{(1+a_1)(1+a_2)x_1^2}\ln\Big(\frac{1 - \tilde{u}}{(1+rx_0-x1)^2}\Big)-\frac{1 - \tilde{u}}{x_1^2}+\frac{(1+rx_0-x_1)^2}{x_1^2} = \alpha
$$
Мы знаем, что $\tilde{u}_\alpha (x)$ - корень уравнения, следовательно, $(\tilde{u}_\alpha (x))' = 0$\\

\vspace{1pc}
Продифференцируем тогда $F(x,\tilde{u}_\alpha(x_1))$ как неявную функцию по $x_1$.\\
\vspace{1pc}
$$
-\frac{2}{x_1^3}\left( (1-\tilde{u}_\alpha(x_1) )\cdot \ln \frac{1-\tilde{u}_\alpha(x_1)}{(1+r)^2 \cdot (1-x_1)^2} - (1-\tilde{u}_\alpha(x_1)) + (1+r)^2 \cdot (1-x_1)^2 \right) -
$$
$$
\frac{1}{x_1^2}\left(  (1-\tilde{u}_\alpha(x_1) )\cdot \ln \frac{1-\tilde{u}_\alpha(x_1)}{(1+r)^2 \cdot (1-x_1)^2} - (1-\tilde{u}_\alpha(x_1)) + (1+r)^2 \cdot (1-x_1)^2  \right)^\prime  = 0 \Leftrightarrow
$$
\vspace{2pc}
Так как из первой формулы можно получить, что $ (1-\tilde{u}_\alpha(x_1)) \cdot \ln \frac{1-\tilde{u}_\alpha(x_1)}{(1+r)^2 \cdot (1-x_1)^2} - (1-\tilde{u}_\alpha(x_1)) + (1+r)^2 \cdot (1-x_1)^2 = (1-\alpha) c x_1^2$, то:\\

\vspace{2pc}
$$
-\frac{2}{x_1^3} (1-\alpha)cx_1^2 - \frac{1}{x_1^2}\left(-\frac{2 (1-\tilde{u}_\alpha(x_1))}{x_1-1} - 2(1+r)^2(1-x_1)\right)=0 \Leftrightarrow
$$

$$
2x_1(1-\alpha)c - \frac{2 (1-\tilde{u}_\alpha(x_1))}{x_1-1} - 2(1+r)^2(1-x_1)=0 \Leftrightarrow
$$
$$
\tilde{u}_\alpha(x_1) = (x_1-1)\left((1-\alpha)x_1c - (1-x_1)(1+r)^2 \right)+1
$$

Ну и, соответственно, подставляем это в
$$
F(x,\tilde{u}_\alpha(x_1)) = \alpha \Leftrightarrow
$$
$$
\Leftrightarrow 1-\frac{1}{c \cdot x_1^2}\left[ (1-\tilde{u}_\alpha(x_1)) \cdot \ln \frac{1-\tilde{u}_\alpha(x_1)}{(1+r)^2 \cdot (1-x_1)^2} - (1 - \tilde{u}_\alpha(x_1)) +(1+r)^2 \cdot (1-x_1)^2 \right] = \alpha
$$

\centering\subsection{Проверка на данных}
\vspace{2pc}
\flushleft

Проверим нашу модель на конкретных числах.\\
Пусть:\\
r = 0.05\\
$\alpha$ = 0.95\\
$\xi_1$ = 0.1\\
$\xi_2$ = 0.05

Тогда, решая нашу задачу в системе Maple, получаем следующий корень уравнения $\tilde{F} (x,\tilde{u}_\alpha(x_1)) = \alpha$:\\
$x_1^*$ = 0.000006648685011.\\
В таком случае, значение $\tilde{u}_{0.95}(x_1^*) = 0.051835210147$

\centering\section{Рассмотрение 3-х периодной модели}

\flushleft



Рассмотрим задачу $\tilde{u}_\alpha(x) -> \min\limits_x$\\
\vspace{1em}
Пусть:
$$
-1\le \xi_1 \le a_1
$$
$$
-1\le \xi_2 \le a_2
$$
$$
-1\le \xi_3 \le a_3
$$
\vspace{1em}
Где $\xi_1,\xi_2,\xi_3$ - значения рисковых бумаг.\\

Введем функцию: $\tilde{F}(x,\tilde{u}) = P(L_3 \le \tilde{u}) = P(1-(1+rx_0+\xi_1 x_1)(1+rx_0+\xi_2 x_1)(1+rx_0+\xi_3 x_1)\le \tilde{u})$\\
\vspace{0.5em}
1. Если $x_1 = 0$, то:\\
$$
\tilde{F}(x,\tilde{u}) =
\left\{\begin{matrix}
1, &(1+r)^3 \ge \tilde{u}-1 \\
0, &(1+r)^3 < \tilde{u}-1,
\end{matrix}\right.
$$
2. Если $x_1 > 0$, то\\



\vspace{0.5em}
Аналогично пункту 5:\\

\vspace{0.5em}
Если $\xi_1 = \xi_1^*$, то $\xi_2 = -1,\xi_3 = -1 $.\\
Таким образом,
$$
\xi_1^* =\xi_2^*=\xi_3^* =\frac{\frac{1-\tilde{u}}{(1+rx_0-x_1)^2} -1 -rx_0}{x_1} = \frac{ 1-\tilde{u} }{ x_1(1+rx_0-x_1)^2 } - \frac{1 -rx_0}{x_1}
$$

$\tilde{F}(x,\tilde{u}) = P(L_3 \le \tilde{u})= \frac{mes(S)}{mes} = 1 - \frac{mes(S^-)}{mes}$


$$
mes(S^-) = \int_{-1}^{\xi_1^*} \int_{-1}^{\xi_2^*} \Big(\frac{1}{x_1} \big(\frac{1-\tilde{u}}{(1+rx_0+\xi_1 x_1)(1+rx_0+\xi_2 x_1)} -1 -rx_0\big)+1\Big)d\xi_2 d\xi_1 =
$$
$$
= \int_{-1}^{\xi_1^*}( \frac{1-\tilde{u}}{x_1(1+rx_0+\xi_1 x_1)}\int_{-1}^{\xi_2^*}\frac{1}{1+rx_0+\xi_2 x_1}d\xi_2 - \int_{-1}^{\xi_2^*}\frac{1+rx_0+ x_1}{x_1}d\xi_2) d\xi_1 =
$$
$$
= \int_{-1}^{\xi_1^*} \frac{1-\tilde{u}}{x_1(1+rx_0+\xi_1 x_1)}\frac{1}{x_1}\ln(\frac{1+rx_0+\xi_2^* x_1}{1+rx_0-x_2}) - \frac{1+rx_0+ x_1}{x_1}(\xi_2^*+1) d\xi_1 =
$$

$$
= \frac{1-\tilde{u}}{x_1^2}\ln(\frac{1+rx_0+\xi_2^* x_1}{1+rx_0-x_1})\int_{-1}^{\xi_1^*}\frac{1}{1+rx_0+\xi_1 x_1}-(\xi_2^*+1)\int_{-1}^{\xi_1^*}\frac{1+rx_0+ x_1}{x_1}d\xi_2 =
$$

$$
= \frac{1-\tilde{u}}{x_1^3}\ln(\frac{1+rx_0+\xi_2^* x_1}{1+rx_0-x_1})\ln(\frac{1+rx_0+\xi_1^* x_1}{1+rx_0-x_1})-(\xi_1^*+1)(\xi_2^*+1)\frac{1+rx_0+x_1}{x_1}
$$

Так как $\xi_1^* = \xi_2^* = \frac{ 1-\tilde{u} }{ x_1(1+rx_0-x_1)^2 } - \frac{1 -rx_0}{x_1}$ то

$$
mes(S^-) = \frac{1-\tilde{u}}{x_1^3} ln^{2}\Big(\frac{1-\tilde{u}}{((1-x_1)(1+r))^3}\Big)-
$$
$$
-\frac{(1-x_1)(1+r)}{x_1}\Bigg(\frac{\frac{1-\tilde{u}}{((1-x_1)(1+r))^{2}} - ((1-x_1)(1+r))}{x_1}\Bigg)^{2}
$$

То есть, необходимо минимизировать по $x_1$ значение $\tilde{u}_\alpha(x_1)$ для выражения $\tilde{F}(x_1,\tilde{u}_\alpha(x_1))=1-\frac{mes(S^-)}{(1+a_1)(1+a_2)(1+a_3)} = \alpha$\\

\centering\subsection{Проверка на данных}
\vspace{2pc}
\flushleft

Проверим нашу модель на конкретных числах.\\
Пусть:\\
r = 0.05\\
$\alpha$ = 0.95\\
$a_1$ = 0.1\\
$a_2$ = 0.08\\
$a_3$ = 0.05

Тогда, решая нашу задачу в системе Maple, получаем следующий корень уравнения $F (x,\tilde{u}_\alpha(x_1)) = \alpha$:\\
$x_1^*$ = 0.245100772079.\\
В таком случае, значение $\tilde{u}_{0.95}(x_1^*) = 0.290682574631$



\centering\section{Обобщение модели}
\flushleft
%$$
%\lim_{n\rightarrow\infty} a_n=\left\{ \begin{matrix} +\infty,\ d>0 \\ -\infty,\ d<0  \\ a_1,\ d=0 \end{matrix} \right.
%$$

Теперь представим $n$-периодный случай.\\
Вспомним вид уже знакомой нам по предыдущим главам функцию доходности $X_n$:
$$
X_n = X_0 \cdot (1+rx_0+\xi_1 x_1) \cdot (1+rx_0+\xi_2 x_1) \cdot ... \cdot (1+rx_0+\xi_n x_1)
$$
Функция потерь тогда равна (с учетом того, что $X_0 = 1$:\\
$$
L_n = 1 - (1+rx_0+\xi_1 x_1) \cdot (1+rx_0+\xi_2 x_1) \cdot ... \cdot (1+rx_0+\xi_n x_1)
$$

Эта функция выражает доходность портфеля после $n$ периодов.\\
В таком случае, функция вероятности выглядит следующим образом:\\
$$
\tilde{F}(x,\tilde{u}) = P(L_n \le \tilde{u})
$$\
Будем считать, что вероятность $\alpha$ задана.\\
Тогда,\\
$$
VaR: \tilde{u}_\alpha (x) = \min\limits_{x} \{\tilde{u}|F(x,\tilde{u})\ge \alpha\}
$$

Наша задача - минимизировать $\tilde{u}_\alpha (x)$.\\
Аналогично предыдущим пунктам, можно построить $n$-мерный график, где точки пересечения с осями выглядят следующим образом:
$$
1-\tilde{u} = (1+rx_0 + \xi^*_1 x_1) \cdot (1+rx_0- x_1) \cdot ... \cdot (1+rx_0- x_1) \Longleftrightarrow
$$

$$
1-\tilde{u} = (1+rx_0 + \xi^*_1 x_1)  (1+rx_0- x_1)^{n-1} \Longleftrightarrow
$$

$$
\xi^*_1 = \frac{1-\tilde{u}}{x_1(1+rx_0- x_1)^{n-1}} -\frac{1 - rx_0}{x_1}
$$

$$
\tilde{F}(x,\tilde{u})=P(L_n \le \tilde{u}) = \frac{mes(S)}{mes} = 1 - \frac{mes(S^-)}{mes}
$$


$$
mes(S^-) = \int\displaylimits_{-1}^{\xi_1^*} \cdot ... \cdot   \int\displaylimits_{-1}^{\xi_{n-1}^*} (\frac{1-\tilde{u}}{x_1} (\frac{1-\tilde{u}}{(1+rx_0+\xi_1 x_1)\cdot ... \cdot (1+rx_0+\xi_2 x_1)k} -1 -rx_0)+1 )d\xi_2 d\xi_1 =
$$
Производим выкладки, аналогичные предыдущим пунктам, и получаем:

$$
= \frac{1-\tilde{u}}{x_1^n} ln^{n-1}\Bigg(\frac{1-\tilde{u}}{(1+rx_0-x_1)^n}\Bigg)-\frac{1+rx_0-x_1}{x_1}\Bigg(\frac{\frac{1-\tilde{u}}{(1+rx_0- x_1)^{n-1}} - (1 + rx_0-x_1)}{x_1}\Bigg)^{n-1}
$$
$$
= \frac{1-\tilde{u}}{x_1^n} ln^{n-1}\Bigg(\frac{1-\tilde{u}}{((1-x_1)(1+r))^n}\Bigg)-\frac{(1-x_1)(1+r)}{x_1}\Bigg(\frac{\frac{1-\tilde{u}}{((1-x_1)(1+r))^{n-1}} - ((1-x_1)(1+r))}{x_1}\Bigg)^{n-1} =
$$
То есть, в случае $n$ периодов, необходимо минимизировать по $x_1$ значение $\tilde{u}_\alpha(x_1)$ для выражения
$$
F(x_1,\tilde{u}_\alpha(x_1))=1-\frac{mes(S^-)}{(1+a_1)\cdot ... \cdot(1+a_n)} = \alpha
$$
\newpage
\centering\section{Заключение}
\flushleft
В первой части работы (главы 2 - 5) сначала были введены определения различных мер риска (VaR, TCE, ES, TM). Далее, были показаны и обоснованы многие их свойства, в частности, когерентность.\\
Во второй части работы (главы 6-8) была поставлена задача оптимизации портфеля, при условии, что соотношение рисковых и безрисковых задается единожды при формировании портфеля и далее не меняются.\\
На основе условий задачи было выведено выражение, при минимизации которого можно получить соотношение рисковых и безрисковых бумаг, которое обеспечивает минимальный уровень потерь,
при определенном уровне уверенности.\\
Для двух- и трех- периодных моделей были проведены вычисления и получены результаты, на основе конкретных значений.


\newpage
\centering\section{Список литературы}
\begin{thebibliography}{1}
\bibitem{one} Rockafellar R.T, Uryasev S. Conditional value-at-risk for general loss distribution. \newblock // Journal of Banking $\&$ Finance. 2002. V. 1. No. 26. P. 1443-1471.
\bibitem{two} Rockafellar R.T, Uryasev S. Optimization of conditional value-at-risk. \newblock // Journal of Risk. 2000. V. 2. No. 3. P. 21-41.
\bibitem{three} Кибзун А.И., Кузнецов Е.А. Сравнение критериев VaR и CVaR.\newblock // Автоматика и телемеханика 2003. No.7. С. 153-165.
\bibitem{four} Кибзун А.И., Кузнецов Е.А. Алгоритм решения обобщенной задачи Марковица.\newblock //  Автоматика и телемеханика. 2011, No.2, 77-92.
\bibitem{five} Кан Ю.С., Кибзун А.И. Задачи стохастического программирования с вероятностными критериями.\newblock // М.: Физматлит, 2009.
\bibitem{six} Artzner, P., Delbaen, F., Eber, J.-M., Heath, D. Coherent measures of risk.\newblock // Mathematical Finance. 1999, V. 9 No. 3, P. 203-228.
\bibitem{seven} Uryasev, S. Conditional Value-at-Risk: Optimization Algorithms and Applications.\newblock // Financial Engineering News. 2000. V. 2, No. 3.
\bibitem{eight} Carlo Acerbi, Dirk Tasche On the coherence of Expected Shortfall.\newblock // Journal of Banking $\&$ Finance. 2002, V. 26, No. 1. P. 1487-1503.
\bibitem{nine}  H. Markowitz.  Portfolio selection.\newblock // Journal of Finance. 1952, V. 7, No. 1. P. 77-91.
\end{thebibliography}



\end{document}
